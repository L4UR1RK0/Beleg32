% Sektion
\section{Sektion}

% Subsektion
\subsection{Untersektion}
Ich bin die Untersekton der Sektion

% Liste
\begin{itemize}
    \item Hallo,
    \item Ich bin eine Liste,
    \item und habe 3 Punkte.
\end{itemize}

% Subsubsektion
\subsubsection{Unteruntersektion}
Ich bin eine Untersektion der Untersektion

% Bild einfügen 
\begin{bafigure}
    [placement = h!, source = https://google.com]
    {Bildunterschrift}        
    \includegraphics[width=10cm]{./graphics/ba.png}    
\end{bafigure}

% Formel
\begin{equation*}
    P(A|B) = \frac{P(B|A) * P(A)}{P(B)}
\end{equation*}
Die verbaliserte Formel des Satz von Bayes lautet:
\begin{equation*}
    A~posteriori = \frac{Likelihood * A~priori}{Evidence}
\end{equation*}

% Beispiel für mathematische Ausdrücle in Text
% \textit{} = kursiver Text
Die Wahrscheinlichkeit $P(A|B)$, dass Event $A$ eintritt, wenn Event $B$ bereits eingetreten ist,
wird als \textit{a posteriori} bezeichnet. Die Likelihood $P(B|A)$ ist die Wahrscheinlichkeit,
dass Event $B$ eintritt, wenn Event $A$ bereits eingetreten ist.
Die \textit{a priori} $P(A)$ ist die Wahrscheinlichkeit ohne zusätzliche Informationen,
dass Event $A$ eintritt. Die Evidence $P(B)$ ist die Wahrscheinlichkeit,
dass Event $B$ eintritt.



$\vec{x}$ %Vektor

Wort $x_i$ in einem Dokument der Klasse $y$ vorkommt, dar.\vglcite{manning2008introduction} %Zitat mit Zitatreferenz in Klammern


% komplexere Formeln
\begin{equation}
    \operatorname{log}(P(y|x_1, ..., x_n))\propto{}\operatorname{log}(P(y)*\prod_{i=1}^{n}P(x_i|y)^{t_i})
\end{equation}

Durch das Anwenden der Logarithmus-Rechenregeln, $\operatorname{log}(x^n) = n*\operatorname{log}(x)$ und $\operatorname{log}(x*y)=\operatorname{log}(x)+\operatorname{log}(y)$, ergibt sich folgende Formel:

\begin{equation}
    \operatorname{log}(P(y|x_1, ..., x_n))\propto{}\operatorname{log}(P(y))+\sum_{i=1}^{n}{t_i}*\operatorname{log}(P(x_i|y))
\end{equation}

\clearpage % Rest der Seite frei lassen